\documentclass[dvipdfmx]{jsarticle}

\usepackage{amsmath, amssymb, amsthm}

\newtheorem{defi}{定義}
\newtheorem{them}{定理}
\newtheorem{prop}[them]{命題}
\newtheorem{lemm}[them]{補題}
\newtheorem{cor}[them]{系}
\newtheorem{exam}{例}

\begin{document}

  \begin{them}
    自然数の二乗の逆数の和は次の実数に収束する:
    \[
      1 + \frac{1}{2^2} + \frac{1}{3^2} + \frac{1}{4^2} + \cdots = \frac{\pi^2}{6}
    \]
  \end{them}

  \begin{them}
    複素数$s$に対し, 無限級数
    \[
      1 + \frac{1}{2^s} + \frac{1}{3^s} + \frac{1}{4^s} + \cdots
    \]
    は$\operatorname{Re}(s)>1$で収束し, $\mathbb{C}\setminus\{1\}$へ有理型に解析接続できる. この函数を$\zeta(s)$で表したとき, $s=-1$での値は
    \[
      \zeta(-1) = -\frac{1}{12}
    \]
    となる.
  \end{them}


\end{document}